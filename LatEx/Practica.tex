\documentclass{article}
\usepackage[utf8]{inputenc}
\usepackage[spanish]{babel}

\begin{document}


\section{Introducción}
\subsection*{Planteamiento del problema}
El problema que se aborda es la falta de persistencia de datos en la aplicación.
La aplicación guarda los datos en la memoria mientras está funcionando, pero se pierde al reiniciar la aplicación.
Se necesita una solución que garantice lo siguiente:
\begin{itemize}
    \item La información de cada item se almacena de forma permanente.
    \item Los usuarios pueden recuperar, actualizar o eliminar estos items en cualquier momento.
    \item La aplicación mantenga un registro organizado y estructurado de todos los paquetes.
\end{itemize}

\subsection*{Propuesta de solución}
La solución propuesta es el desarrollo de una API RESTful utilizando FastAPI para la lógica de negocio y SQLModel para la gestión de la base de datos , permitiendo así la persistencia de datos.
Componentes clave de la solución:
\begin{itemize}
    \item \textbf{Framework Web (FastAPI):} Se utiliza para construir la interfaz de la API, definiendo las rutas (URLs) que permitirán a los usuarios interactuar con la información de los paquetes.} Un framework web moderno y rápido para construir APIs con Python 3.7+ basado en estándares como OpenAPI y JSON Schema.
    \item \textbf{SQLModel:} Se emplea para definir la estructura de la información (qué es un "ítem", qué campos tiene, y cuál es su identificador único).
    \item \textbf{Base de datos SQLite:} Se utiliza un motor de base de datos SQLite (database.db) para almacenar los datos en un archivo físico. De esta manera, la información perdura aunque la aplicación se detenga o se reinicie.
    \item \textbf{SQLModel Engine y Session:} Se configura un motor de conexión (engine) y un sistema de sesiones (SessionDep) para establecer una comunicación eficiente y segura entre la aplicación y el archivo de la base de datos.
\end{itemize}

\end{document}